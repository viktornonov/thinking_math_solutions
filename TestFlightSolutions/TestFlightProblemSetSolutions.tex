\documentclass{article}

\title{Test Flight Problem Set}
\author{Viktor Nonov}
\date{May 4, 2018}
\usepackage[cm]{fullpage}
\usepackage{enumerate}
\usepackage{amsmath}
\usepackage{amssymb}

\begin{document}

\section*{Problem 1}
Some branches of mathematics include 0 in $\mathbb{N}$, others do not, so I'm going to check if the expression is true or false in both cases:\\
1) Let's say $0 \in \mathbb{N}$\\
then it is true for m = 4 and n = 0:\\
$3*4 + 5*0 = 12$\\
\\
2) Let's say $0 \not\in \mathbb{N}$\\
In this case it's false:\\
Using proof by cases, I'm going to check values for m and n to show that there are no m,n that satisfy the equation $3m + 5n = 12$.\\
Let m = 1, n = 1,\\
$3*1 + 5*1 = 8 < 12$\\
\\
Let m = 2, n = 1,\\
$3*2 + 5*1 = 11 < 12$\\
\\
Let $m \geq 3$, n = 1,\\
$3*3 + 5*1 \geq 14 > 12$\\
\\
Let m = 1, $n \geq 2$,\\
$3*1 + 5n \geq 13 > 12$\\
\\
Those cases cover all values for m and n in $\mathbb{N}$, so $(\exists m \in \mathbb{N})(\exists n \in \mathbb{N})[3m + 5n = 12]$ is false.

\section*{Problem 2}
The sum of any five consecutive integers is divisble by 5.\\
The statement is true.\\
Proof:\\
Let's denote the first integer as k and k can be any integer. Then the next integers in the sequence would be k + 1, k + 2, k + 3, k + 4.\\
Summing all those integers:\\
k + (k + 1) + (k + 2) + (k + 3) + (k + 4)\\
= 5k + 10\\
= 5(k + 2), which is divisble by 5 without a remainder. QED\\

\section*{Problem 3}
$(\forall n \in \mathbb{Z})[n^2 + n + 1$ is odd$]$ is true.\\
Proof:\\
$n^2 + n + 1 = n(n+1) + 1$\\
$n(n+1)$ are product of two consequtive integers.\\
I'm gonna show that this product is an even number.\\
\\
if n is even, then we can denote it as 2k. Substituting:\\
$2k(2k + 1)$, which shows that the product is even.\\
if n is odd, then we can denote it as 2k. Substituting:\\
$(2k+1)(2k+2) = (2k+1)2(k+1)$, which is also even.\\
With this is proved that product of two consequtive integers is an even number.\\
\\
so after showing that $n(n+1)$ is even, I denote it with 2k.\\
$n(n+1) + 1 = 2k + 1$, which is and odd number, which proves that the original statement is true. QED.

\section*{Problem 4}
Prove that every odd natural number is of one of the forms 4n+1 or 4n+3, where n is an integer.\\
Proof:\\
The definition of an odd number is $2k + 1$ for every k.\\
We have two cases - when k is even and when k is odd. These two cases cover all values k in $\mathbb{N}$ that we care about:\\
1. Let's assume k is even and represent it as 2n\\
$2*2n + 1 = 4n + 1$, which is one of the forms above.\\
\\
2. Let's assume k is odd and represent it as 2n + 1\\
$2*(2n + 1) + 1 = 4n + 3$, which is the other form that we needed. With this the proof is concluded.\\

\section*{Problem 5}
Prove that for any integer n, at least one of the integers n; n + 2; n + 4 is divisible by 3.\\
Proof:\\
Representing this with quantifiers:\\
$(\forall n \in \mathbb{Z})[3 | n \vee 3 | (n + 2) \vee 3 | (n + 4)]$\\
Using proof by cases, we have 3 distinct representations of n with respect to divisibility by 3:\\
$n = 3k$\\
$n = 3k + 1$\\
$n = 3k + 2$\\
\\
1. Assume n = 3k, then n is divisible by 3.\\
\\
2. Assume n = 3k + 1, then n + 2 = 3k + 3 = 3(k + 1) is divisible by 3.\\
\\
3. Assume n = 3k + 2, then n + 4 = 3k + 6 = 3(k + 2) is divisible by 3.\\
QED.

\section*{Problem 6}
Prove that the only prime triple (i.e. three primes, each 2 from the next) is 3, 5, 7.\\
Proof:\\
The prime numbers $> 2$ are always odd, so let's represent a prime triple as:\\
2k + 1\\
2k + 3\\
2k + 5\\
For k = 1, we have the triple 3, 5, 7. Throught trying with values for $k > 1$, you can see that one of representations is divided by 3, so using proof by cases I'll show that always one of the presentations above are divisble by 3.\\
In order to list all the cases I'll list all possible representations of k with respect of divisibility by 3:\\
1. k = 3n\\
2. k = 3n + 1\\
3. k = 3n + 2\\
\\
1. Assume k = 3n\\
Using $2k + 3 = 2*3n + 3 = 3(2n + 1)$ is shown that this representation is divisible by 3.\\
\\
2. Assume k = 3n + 1\\
Using $2k + 1 = 2(3n + 1) + 1 = 6n + 3 = 3(2n+3)$, which shows that this representation is divisible by 3.\\
\\
3. Assume k = 3n + 2\\
Using $2k + 5 = 2(3n+2) + 5 = 6n + 9 = 3(2n + 3)$, which shows that this representaiton is divisble by 3.\\
QED.

\section*{Problem 7}
Prove that for any natural number n,\\
$2 + 2^2 + 2^3 + .. + 2^n = 2^{n+1}-2$\\
Proof:\\
Using proof by mathematical induction:\\
Base case: for n = 1 is true, because $2 = 2^2 - 2$.\\
Induction hypothesis:\\
Assume the expression is true for n = k.\\
$2 + 2^2 + 2^3 + .. + 2^k = 2^{k+1} - 2$\\
Let n = k + 1, then we need to prove that\\
$2 + 2^2 + 2^3 + .. + 2^k + 2^{k+1} = 2^{k+2} - 2$\\
Getting left side:\\
$2 + 2^2 + 2^3 + .. + 2^k + 2^{k+1}$\\
$2^{k+1} - 2 + 2^{k+1}$ (substituting by the induction hypothesis)\\
$2^{k+2}-2$ (by algebratic substitution)\\
by the principle of math induction the original statement is proved.

\section*{Problem 8}
Prove (from the defnition of a limit of a sequence) that if the sequence $\{a_n\}_{n=1}^\infty$ tends to limit L as $n \rightarrow \infty$, then for any fixed number $M > 0$, the sequence $\{Ma_n\}_{n=1}^\infty$  tends to the limit ML.\\
Proof:\\
From the definition of a limit:
$\{a_n\}_{n=1}^\infty$ tends to limit L as $n \rightarrow \infty$ $\Leftrightarrow |a_m - L| < \varepsilon$, where $m \in \mathbb{N}$ and it is arbitrary large.\\
$|a_m - L| < \varepsilon$ /*$M > 0$\\
$M|a_m - L| < M*\varepsilon$\\
$|Ma_m - ML| < M*\varepsilon$ (1)\\
\\
Again from the definition of a limit the sequence $\{Ma_n\}_{n=1}^\infty$  tends to the limit ML as $n \rightarrow \infty$ means that:\\
$(\forall \varepsilon)(\exists n \in \mathbb{N})(\forall m \geq n)[|Ma_m - ML| < \varepsilon]$. Given that definition (1) satisfies it, so $\{Ma_n\}_{n=1}^\infty$  tends to the limit ML if  $\{a_n\}_{n=1}^\infty$ tends to limit L. QED.

\section*{Problem 9}
Example of family of such intervals is the intervals in the form $(0, \frac{1}{n})$.\\
for n = 1, the interval is (0, 1)\\
for n = 2, the interval is $(0, \frac{1}{2})$\\
for n = 3, the interval is $(0, \frac{1}{3})$\\
so the least upper bound is decresing with increasing of n\\
This shows that $A_{n+1} \subset A_{n}$.\\
with $n \rightarrow \infty$, $\frac{1}{n}$ is getting closer to the limit 0 (by the definition of a limit).\\
so eventually the interval will tend to (0,0), which is $\emptyset$ and when there is intersection of all intervals in the family $\cap_{n=1}^\infty(0, \frac{1}{n})$, the it will result in empty set too. QED\\

\section*{Problem 10}
Example of family of such intervals is the intervals in the form $[1, \frac{n+1}{n})$.
for n = 1, the interval is [1, 2)\\
for n = 2, the interval is $[1, \frac{3}{2})$\\
for n = 3, the interval is $[1, \frac{4}{3})$\\
so the least upper bound is decresing with increasing of n\\
This shows that $A_{n+1} \subset A_{n}$.\\
with $n \rightarrow \infty$, $\frac{n+1}{n}$ is getting closer to the limit 1 (by the definition of a limit).\\
so eventually the interval will tend to [1,1), which is set of one element $\{1\}$ and when there is intersection of all intervals in the family $\cap_{n=1}^\infty(0, \frac{n+1}{n})$, the it will result in set with one element that is contained by every iterval in the family - $\{1\}$. QED\\

\end{document}
