\documentclass{article}
\title{Assignment 4}
\author{Viktor Nonov}
\date{March 9, 2018}
\usepackage[cm]{fullpage}
\usepackage{enumerate}
\usepackage{amsmath}
\usepackage{xpatch}

% used to reset the couters in differnet parts
\makeatletter
\@addtoreset{section}{part}
\makeatother

\begin{document}
\part{Assignment 4 Solutions}

\section{}
By defnition $\phi \Leftrightarrow \psi$ when $\phi \implies \psi$ and $\psi \implies \phi$, so the truth table should like this:\\
\begin{tabular}{ c c c c c }
  $\phi$ & $\psi$ & $\phi \implies \psi$ & $\psi \implies \phi$ & $(\phi \implies \psi) \wedge (\psi \implies \phi)$ \\
  T      & T      & T                    & T                    & T                                                  \\
  T      & F      & F                    & T                    & F                                                  \\
  F      & T      & T                    & F                    & F                                                  \\
  F      & F      & T                    & T                    & T                                                  \\
\end{tabular}

\section{}
\begin{tabular}{ c c c c c }
  $\phi$ & $\psi$ & $\phi \implies \psi$ & $\neg \phi \vee \psi$ & $(\phi \implies \psi) \Leftrightarrow (\neg \phi \vee \psi)$ \\
  T      & T      & T                    & T                    & T                                                  \\
  T      & F      & F                    & F                    & T                                                 \\
  F      & T      & T                    & T                    & T                                                 \\
  F      & F      & T                    & T                    & T                                                  \\
\end{tabular} \\

$(\phi \implies \psi) \Leftrightarrow (\neg \phi \vee \psi)$ is tautology. \\

\section{}
\begin{tabular}{ c c c c c }
  $\phi$ & $\psi$ & $\phi \not\Rightarrow \psi$ & $\phi \wedge \neg\psi$ & $(\phi \not\Rightarrow \psi) \Leftrightarrow (\phi \wedge \neg\psi)$ \\
  T      & T      & F                    & F                    & T                                                  \\
  T      & F      & T                    & T                    & T                                                 \\
  F      & T      & F                    & F                    & T                                                 \\
  F      & F      & F                    & F                    & T                                                  \\
\end{tabular} \\

\section{}
a. \\
\begin{tabular}{ c c c c c }
  $\phi$ & $\psi$ & $\phi \implies \psi$ & $\phi \wedge (\phi \implies \psi)$ & $(\phi \wedge (\phi \implies \psi)) \implies \psi$ \\
  T      & T      & T                    & T                    & T                                                  \\
  T      & F      & F                    & F                    & T                                                 \\
  F      & T      & T                    & F                    & T                                                 \\
  F      & F      & T                    & F                    & T                                                  \\
\end{tabular}
\\
b. \\
The last column shows that $(\phi \wedge (\phi \implies \psi)) \implies \psi$ is tautology, which means that it is always True, no matter what the valuation of the propositions is. This means that the it is going to be possible to find $\psi$ in every case. \\n
Another way would be to write a proof that  $(\phi \wedge (\phi \implies \psi)) \implies \psi$ is not affected by the valuation of its propositions:\\
$(\phi \wedge (\phi \implies \psi)) \implies \psi$ \\
$(\phi \wedge (\neg\phi \vee \psi)) \implies \psi$ \\
$\neg(\phi \wedge (\neg\phi \vee \psi)) \vee \psi$ \\
$\neg((\phi \wedge \neg\phi) \vee (\phi \wedge \psi)) \vee \psi$ \\
$\neg((F) \vee (\phi \wedge \psi)) \vee \psi$ \\
$\neg(\phi \wedge \psi) \vee \psi$ \\
$(\neg\phi \vee \neg\psi) \vee \psi$ \\
$\neg\phi \vee \neg\psi \vee \psi$ \\
$\neg\phi \vee T$   \\
$T$

\section{}
Prove that $\neg (\phi \vee \psi) \Leftrightarrow \neg\phi \wedge \neg\psi$: \\
1. If $\phi \vee \psi$ is true, then at least one of $\phi$ or $\psi$ is true \\
2. But $\neg(\phi \vee \psi)$ = true shows that, 1. is not the case. \\
3. Therefore both $\phi$ and $\psi$ must be false. \\
4. By the definition of conjunction it means that 3. can be represented as $\neg\phi \wedge \neg\psi$, which proves the equivalence. \\

\section{}
a) 34159 is a prime number. \\
It's simple statement P, so the negation would be $\neg P$ \\
34159 is not a prime number \\
\\
b) Roses are red and violets are blue. \\
a = Roses are red. \\
b = Violets are blue. \\
The statement is represented by: $a \wedge b$.\\
The negation will be: $\neg a \vee \neg b$ \\
Possible denials are: \\
Roses are not read or violets are not blue. \\
Either roses aren't red or violets aren't blue or both. \\
\\
c) if there are no hamburgers, I will have a hotdog. \\
there are no hamburgers = a \\
I will have a hotdog = b \\
so the statement can be captured with a conditional: $a \implies b$\\
So the denial of statement is when the statement is negated: $\neg(a \implies b)$ \\
Converting and using the DeMorgan laws: $\neg(\neg a \vee b) \Leftrightarrow (a \wedge \neg b)$\\
Which means: \\
There are no hamburgers and I won't have a hotdog. \\
\\
d) Fred will go, but he will not play. \\
Fred will go = a \\
he will play = b \\
Statement is represented by $a \wedge \neg b$\\
The denial of statement would be $\neg a \vee b$, which is the conditional $a \implies b$, so the statement would be: \\
If Fred goes, then he will play. \\
\\
e) The number x is either negative  or greater than 10. \\
The statement is: $x < 0 \vee x > 10$. \\
The denial of statement is: $\neg (x < 0 \vee x > 10) \Leftrightarrow x >= 0 \wedge x <= 10$ \\
The number is not negative and less or equal to 10.\\
\\
f) We will win the first or the second game. \\
win the first game = a \\
win the second game = b \\
The statement is: $a \vee b$ \\
The denial of statement is: $\neg a \wedge \neg b$ \\
We won't win the first nor the second game. \\

\section{}
Prove by constructing the truth table: \\
\begin{tabular}{ c c c c c c }
  $\phi$ & $\psi$ & $\neg\phi$ & $\neg\psi$ & $\phi \Leftrightarrow \psi$ & $\neg \phi \Leftrightarrow \neg \psi$ \\
  T      & T      & F          & F          & T                           & T                                       \\
  T      & F      & F          & T          & F                           & F                                        \\
  F      & T      & T          & F          & F                           & F                                        \\
  F      & F      & T          & T          & T                           & T                                        \\
\end{tabular} \\
The truth table shows that they are equivalent. \\
\\
Logical proof: \\
1. If $\phi \Leftrightarrow \psi$ is true, then both $\phi$ and $\psi$ are either true or false. \\
2. In order for $\neg\phi \Leftrightarrow \neg\psi$ to be true, then $\neg\psi$ and $\neg\phi$ should be either true or false. \\
3. If $\neg\psi$ and $\neg\phi$ are T, then $\psi$ and $\phi$ are F. The oppisite is also true. Which means that it is the same as 1.\\

\section{}
a.\\
\begin{tabular}{ c c c }
  $\phi$ & $\psi$ & $\phi \Leftrightarrow \psi$  \\
  T & T & T \\
  T & F & F \\
  F & T & F \\
  F & F & T \\
\end{tabular} \\
b.\\
\begin{tabular}{ c c c c c }
  $\phi$ & $\psi$ & $\theta$ & $\psi \vee \theta$ & $\phi \implies (\psi \vee \theta)$ \\
  T      & T      & T          & T          & T  \\
  T      & T      & F          & T          & T  \\
  T      & F      & T          & T          & T  \\
  T      & F      & F          & F          & F  \\
  F      & T      & T          & T          & T  \\
  F      & T      & F          & F          & T  \\
  F      & F      & T          & T          & T  \\
  F      & F      & F          & F          & T  \\
\end{tabular} \\

\section{}

\begin{tabular}{ c c c c c c c c }
  $\phi$ & $\psi$ & $\theta$ & $\psi \wedge \theta$ & $\phi \implies \psi$ & $\phi \implies \theta$ & $\phi \implies (\psi \wedge \theta)$ & $(\phi \implies \psi) \wedge (\phi \implies \theta)$\\
  T      & T      & T          & T          & T                           & T             & T & T                          \\
  T      & T      & F          & F          & F                           & F             & F & F                         \\
  T      & F      & T          & T          & T                           & T             & T & T                        \\
  T      & F      & F          & F          & T                           & T             & T & T                         \\
  T      & T      & T          & F          & T                           & F             & F & F                        \\
  T      & T      & F          & F          & F                           & T             & F & F                        \\
  T      & F      & T          & F          & T                           & T             & T & T                         \\
  T      & F      & F          & F          & T                           & T             & T & T                         \\
\end{tabular} \\
This table shows some form of the distributive property: $\phi \implies (\psi \wedge \theta) \Leftrightarrow (\phi \implies \psi) \wedge (\phi \implies \theta)$\\

\section{}
$(\psi \implies (\phi \wedge \theta)) \implies (\phi \implies \psi) \wedge (\phi \implies \theta)$\\
1. if we know $\phi \implies (\psi \wedge \theta)$ then we can get the value of $\psi \wedge \theta$.\\
2. Once we know this value, then we can have the value of $\psi$ and $\theta$.\\
from 1 and 2 follows that we can deduce $\psi$ and $\theta$ from $\phi$ - $(\phi \implies \psi) \wedge (\phi \implies \theta)$ \\
\\
$((\phi \implies \psi) \wedge (\phi \implies \theta)) \implies (\psi \implies (\phi \wedge \theta))$\\
1. we can find the value of $\psi$ from $\phi$ and $\theta$ from $\phi$ \\
2. once we know their values we know the value of $\psi \wedge \theta$ \\
3. from 1 and 2 follows that we can get $\phi$ from $\psi \wedge \theta$ \\
$\psi \implies (\phi \wedge \theta)$ \\

\section{}
\begin{tabular}{ c c c c }
  $\phi$ & $\psi$ & $\phi \implies \psi$ & $\neg \psi \implies \neg \phi$ \\
  T      & T      & T                    & T                              \\
  T      & F      & F                    & F                              \\
  F      & T      & T                    & T                              \\
  F      & F      & T                    & T                              \\
\end{tabular} \\

\section{}
a. if two rectangles does not have the same area then they are not congruent \\
b. if a triangle with sides a, b, c(c is the largest) $a^2 + b^2 \neq c^2$ then it is not right triangle. \\
c. if n is not prime, then $2^n - 1$ is not prime \\
d. if Dollar rise, the Yuan fail.

\section{}
\begin{tabular}{ c c c c }
  $\phi$ & $\psi$ & $\psi \implies \phi$ & $\neg \psi \implies \neg \phi$ \\
  T      & T      & T                    & T                              \\
  T      & F      & T                    & F                              \\
  F      & T      & F                    & T                              \\
  F      & F      & T                    & T                              \\
\end{tabular} \\

\section{}
a. if the two rectangles have the same area then they are congruent \\
b. if triangle with sides a,b,c (c is the largest) and $a^2 + b^2 = c^2$ then it is right angled. \\
c. if n is prime then $2^n - 1$ is prime. \\
d. if the dollar fails, then yuan rises. \\

\part{OPTIONAL PROBLEMS}

\section{}
\begin{tabular}{ c c c c }
  $\phi$ & $\psi$ & $\neg \psi$ & $\neg \psi \implies \phi$ \\
  T      & T      & F           & T                         \\
  T      & F      & T           & T                         \\
  F      & T      & F           & T                         \\
  F      & F      & T           & F                         \\
\end{tabular} \\

\section{}
\begin{tabular}{ c c c }
  $\phi$ & $\psi$ & $\phi \oplus \psi$ \\
  T      & T      & F                  \\
  T      & F      & T                  \\
  F      & T      & T                  \\
  F      & F      & F                  \\
\end{tabular} \\

\section{}
prove by truth table \\
\begin{tabular}{ c c c c c}
  $\phi$ & $\psi$ & $\neg \phi \wedge \psi$ & $\phi \wedge \neg \psi$ & $(\neg \phi \wedge \psi) \vee (\phi \wedge \neg \psi)$ \\
  T      & T      & F                       & F                       & F                                                      \\
  T      & F      & F                       & T                       & T                                                      \\
  F      & T      & T                       & F                       & T                                                      \\
  F      & F      & F                       & F                       & F                                                      \\
\end{tabular} \\

\section{}
a. if the converse is T then the conditional is actually biconditional, so both sides are equivalent - if two rectangles have the same area then they are congruent $\Leftrightarrow$ if two triangles are congruent then they have the same area.\\
b. "if it is Sunday, then I'm not gonna go to work today". The converse "If I'm not gonna go to work today, then it means it's Sunday.", which is not always true.\\
c. every contraposive of a conditional that is true is also true, because they are equivalent. \\
d. conditional and contrapositive are equivalent, so it is not possible to have conditional that is true and contrapositive that is false \\

\section{}

\begin{tabular}{ c c c c }
  $M$  & $N$  & $M x N$ & $M + N$ \\
  T(1) & T(1) & 1       & 0       \\
  T(1) & F(0) & 0       & 1       \\
  F(0) & T(1) & 0       & 1       \\
  F(0) & F(0) & 0       & 0       \\
\end{tabular} \\

\section{}
a. $\wedge$ is x \\
b. $\oplus$ (XOR) corresponds to + \\
c. it does not look like it \\
Let's take - as unary operator, so $-N \equiv \neg N$ \\
we use the fact that + is XOR to check if it works for summing \\
$M + (-N) \Leftrightarrow M \oplus \neg N$\\
\begin{tabular}{ c c c c }
  $M$  & $N$  & $\neg N$ & $M \oplus \neg N \equiv (M - N)$ \\
  T(1) & T(1) & F(0)     & T(1)                             \\
  T(1) & F(0) & T(1)     & F(0)                             \\
  F(0) & T(1) & F(0)     & F(0)                             \\
  F(0) & F(0) & T(1)     & T(1)                             \\
\end{tabular} \\
The truth table shows that for the case 0 - 1 we are getting 1, which in normal arithmetic is -1, so it does not work for normal arithmetic. \\
In mod2 arithmetic though $-1 \equiv 1 (\mod 2)$, because $(-1 \mod 2 = 1)$, so in mod2 arithmetic $\neg \equiv -$\\

\section{}
\begin{tabular}{ c c c c }
  $M$  & $N$  & $M * N$ & $M + N$ \\
  F(1) & F(1) & F(1)    & T(0)    \\
  F(1) & T(0) & T(0)    & F(1)    \\
  T(0) & F(1) & T(0)    & F(1)    \\
  T(0) & T(0) & T(0)    & T(0)    \\
\end{tabular} \\
a. Multiplication (x) becomes disjunction $\vee$ \\
b. Summation (+) becomes inverted XOR which is equivalence $\Leftrightarrow$ \\
c.
\begin{tabular}{ c c c c }
  $M$  & $N$  & $\neg N$ & $M \neg N \equiv (M - N)$ \\
  T(0) & T(0) & F(1)     & T(0)                             \\
  T(0) & F(1) & T(0)     & F(1)                             \\
  F(1) & T(0) & F(1)     & F(1)                             \\
  F(1) & F(1) & T(0)     & T(0)                             \\
\end{tabular} \\
Here the subtraction and $\neg$ can be equivalent in mod2 arithmetic if negation is considered binary connective, but in logic $\neg$ is unary connetive, so I can't think of any useful conclusions here. \\

\section{}
Given:\\
vowel = a \\
not vowel = $\neg a$ \\
odd = b \\
even = $\neg b$ \\

\begin{tabular}{ c c c c c }
  $a$ & $b$ & $\neg a$ & $ \neg b$ & $a \implies b$ \\
  T   & T   & F        & F         & T              \\
  T   & F   & F        & T         & F              \\
  F   & T   & T        & F         & T              \\
  F   & F   & T        & T         & T              \\
\end{tabular} \\
We have the cards: \\
\begin{tabular}{ c c c c  }
  B        & E   & 4        & 7   \\
  $\neg a$ & $a$ & $\neg b$ & $b$ \\
\end{tabular} \\

We can use that $a \implies b \Leftrightarrow \neg b \implies \neg a$, so we know for sure that:\\
below E we should have odd number\\
below 4 we should have not vowel letter\\
we don't care about 7, because the opposite side can vower or not vowel and the implication still is gonna be valid. (1st and 3rd rows) \\
We do not care about B either, because of the reasons above.\\

Looking at the truth table the cards that we need to check are: \\
4 - if it is not vowel then the implication is valid, otherwise it's not. \\
E - if it is odd then the implication is valid, otherwise it's not.

\section{}
Prove that: $m*n = odd number \Leftrightarrow m = odd number \wedge n = odd number$ \\
Let's prove first: \\
$m = odd  \wedge n = odd \implies m*n = odd$ \\
represent a odd number as $2k + 1$, so:\\
$m = 2k_1 + 1$ and $n = 2k_2 + 1$\\
$m*n = (2k_1 + 1)(2k_2+1) = 4k_1k_2 + 2(k_1 + k_2) + 1 = 2(2k_1k_2 + k_1 + k_2) + 1$,\\
which is odd $\forall k_1,k_2 \in N$\\

Now we need to prove that:
$m * n = odd \implies m = odd \wedge n = odd$. \\
using contrapositive: \\
$n = even or m = even \implies m*n = even$ \\
even nubmer can be represented as: $2k$, odd number as: $2k +1$ so \\
1. both are even: \\
if $m = 2k_1$ and $n = 2k_2$, then $2k_1*2k_2 = 4k_1k_2$, which is even\\
2. one even one odd: \\
if $m = 2k_1$ and $n = 2k_2 + 1$, then $2k_1*(2k_2 + 1)$, which is even\\
3. m odd, n even: \\
analogy of 2. \\
so for all the cases when the ancedent is true, the consequent is also true, so the implicaiton is true, which proves the contrapositive, which proves the original $m * n = odd \implies m = odd \wedge n = odd$. \\

\section{}
No, because of the counter example 2 * 3 = 6 - even, when 2 is even and 3 is not even.

\section{}
This questions is similar to question 8, so following the same logic we should check the Beer guy and the under age person.

\section{}
Youth party problem was easier for me, probably because: \\
- I solved the Wason problem first, which is the formal version of the Youth party problem \\
- The youth party problem is a problem that I've faced in real life. \\

\end{document}
