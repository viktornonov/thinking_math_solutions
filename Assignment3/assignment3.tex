\documentclass{article}
\title{Assignment 3}
\author{Viktor Nonov}
\date{March 7, 2018}
\usepackage[cm]{fullpage}
\usepackage{enumerate}
\usepackage{amsmath}

\begin{document}
\section*{Assignment 3 Solutions}

\section{}
Dollar is strong - D, Yuan is strong - Y, New US-China trade agreement signed - T

\begin{enumerate}[a.]
\item $T \implies D \wedge Y$
\item $D \implies \neg Y$
\item $\neg D \implies \neg T$
\item $T \implies (\neg D \wedge Y) \vee (D \wedge \neg Y)$
\item $T \implies \neg D \wedge Y$
\item $T \implies Y \implies \neg D$
\item $T \implies Y \wedge D \implies \neg Y \wedge \neg D$
\item $T \implies (\neg D \wedge Y) \vee (D \wedge \neg Y)$
\end{enumerate}

\section{}
\begin{tabular}{ l l l l l }
  $\phi$ & $\neg \phi$ & $\psi$ & $\phi \implies \psi$ & $\neg \phi \vee \psi$ \\
  T      & F           & T      & T                       & T                     \\
  T      & F           & F      & F                       & F                     \\
  F      & T           & T      & T                       & T                     \\
  F      & T           & F      & T                       & T                     \\
\end{tabular}

\section{}
The conclusion from the truth table above is that the conditional $\phi \implies \psi$ is equivalent to $\neg \phi \vee \psi$.

Example:
Do not copy your homework, or you will fail. $\Leftrightarrow$ If you copy your homework, then you will fail.

\section{}
\begin{tabular}{ l l l l l l }
  $\phi$ & $\psi$ & $\neg \psi$ & $\phi \implies \psi$ & $\phi \not\Rightarrow \psi$ & $\phi \wedge \neg \psi$ \\
  T      & T      & F           & T                    & F                           & F                       \\
  T      & F      & T           & F                    & T                           & T                       \\
  F      & T      & F           & T                    & F                           & F                       \\
  F      & F      & T           & T                    & F                           & F                       \\
\end{tabular}

\section{}
The conclusion from the truth table above is that the condtional $\phi \not\Rightarrow \psi$ is equivalent to $\phi \wedge \neg \psi$.
Which also can be derived by using the DeMorgan law:

$\phi \implies \psi \Leftrightarrow \neg \phi \vee \psi$

$\phi \not\Rightarrow \psi \Leftrightarrow \neg(\phi \implies \psi) \Leftrightarrow \neg(\neg \phi \vee \psi) \Leftrightarrow \phi \wedge \neg \psi$ 

\end{document}
