\documentclass{article}

\title{Division theorem}
\author{Viktor Nonov}
\date{April 23, 2018}
\usepackage[cm]{fullpage}
\usepackage{enumerate}
\usepackage{amsmath}
\usepackage{amssymb}

\begin{document}

\section*{Division Theorem/Euclidean Division}
Let $a,b$ be integers, $b > 0$. Then there exist unique integers q and r such that $a = qb + r$ and $0 \leq r < b$.\\

Proof:\\
Split the problem into proving existance and proving uniqueness.\\
\\
\\
\subsection*{Proof of existance of q and r:}
Expressing r through the given formula: $a = qb + r$. We have: $r = a - qb$. \\
We also know that $r \geq 0$\\
Let's consider the set of all reminders in the form $a - kb \geq 0$, where $k \in \mathbb{Z}$\\
$S = \{a - kb, a' - k'b', a'' - k''b'',...\}$, where all members are non-negative numbers (because we need to have remainders greater or equal to zero $r \geq 0$).\\
If we can prove that this set is non-empty then we can prove that there exist remainder in the form $a - kb \geq 0$ where $b > a - kb \geq 0$.\\
\\
There are such integers:\\
$b > a - kb \geq 0 \implies b > 0$\\
$b > 0 \wedge b \in \mathbb{N} \implies b \geq 1$\\
For example let's take $b = 1$ and $k = -|a|$, so\\
$a - kb \geq 0$\\
$a + |a| \geq 0$, which is true for each a\\
\\
Now we know that the set S has members. We also know that the members are all non-negative, so by WOP (axiom), we can conclude that the set S has a smallest element. (1)\\
Let's say this element is $r = a - qb$.\\
In order to finish the proof of existance we should prove that $r < b$\\
Using proof by contradiction, let's assume $r \geq b$\\
$a - qb \geq b$ (by substituting r)\\
$a - qb - b \geq 0$\\
$a - b(q + 1) \geq 0$\\
this looks like the form $a - bk$, where k = q + 1. It's also non-negative, so it must be a member of the set of reminders S.\\
Let's compare $r = a - bq$ and $a - b(q + 1)$\\
$a - b(q + 1) < a - bq$, because $b(q + 1) > bq$\\
but this is contradiction to (1), so $r < b$\\
\\
This concludes the proof of existance: There exist q and r such that $a = qb + r$ and $0 \leq r < b$.
\\
\\
\subsection*{Proof of uniqueness of q and r:}
To prove the uniqueness we are gonna try to show that there's q' and r', but they are equal to q and r.\\
Let's assume that we can represent $a$ in two ways:\\
$a = bq + r = bq' + r'$\\
$r - r' = bq' - bq$\\
$r - r' = b(q' -q)$\\
\\
From the existance proof we know that $0 \leq r' < b$ and $0 \leq r < b$.\\
Representing $-r'$ using $0 \leq r' < b$, we get $0 \geq -r' > -b$\\
let's sum both inequalitites:\\
$-b < -r' \leq 0$ /+\\
$0 \leq r < b$\\
\\
$-b < r - r' < b$, which means that $|r - r'| < b$\\
$|b(q' - q)| < b$ (move b out of the module because $b > 0$ )\\
$b|q' - q| < b$ dividing by $b \geq 1$\\
$|q' - q| < 1$\\
this means $-1 < q' - q < 1$. Given the fact that both q' and q $\in \mathbb{N}$ the result of their subtraction should also be in $\mathbb{N}$\\
The only way for that to be possible $q' - q = 0 \equiv q' = q$, which proves the uniqueness of q and r.
\end{document}
