\documentclass{article}

\title{Division theorem / Euclidean Division}
\author{Viktor Nonov}
\date{April 23, 2018}
\usepackage[cm]{fullpage}
\usepackage{enumerate}
\usepackage{amsmath}
\usepackage{amssymb}

\begin{document}
\maketitle

Let $a,b$ be integers, $b > 0$. Then there exist unique integers q and r such that $a = qb + r$ and $0 \leq r < b$.\\
In notation:\\
$(\forall a,b \in \mathbb{Z})(b > 0)(\exists! q,r \in \mathbb{Z})[a = bq + r \wedge 0 \leq r < b]$.\\

\subsubsection*{Exploration}
Let's start by exploring the description a bit more and add more details about the members of the division:\\
Reminder is the amount left over after performing division.\\
Let's say you divide: $a / b = q\ (\textrm{remainder}\ r)$, where $a$ can be expressed as $a = bq + r$\\
$q$ is the quotient which is the integer result of the division, so $q \in \mathbb{Z}$.\\
$r$ can be expressed as: $r = a - qb$, where $a, b \in \mathbb{Z}$, $b > 0$ and $q \in \mathbb{Z}$. \\
Also in order for $r$ to be the remainder of divison: $a / b$, then $r$ should be between 0 and $b$: $0 \leq r < b$\\
I'm going to split the proof into two different sections: proving existance and proving uniqueness.

\section*{Proof:}
Split the problem into proving existance and proving uniqueness.\\

\subsection*{Proof of existance of q and r:}
To prove existance of r and q, we should prove them one by one:\\
Let's start with $r$.

\subsubsection*{Proof of existance of r:}
Let's start exploring by trying out some specific examples, which would help construct the general case. \\
For example: $a = 27$ and $b = 5$, then the reminder would be $r = 2$ and quotient $q = 5$\\
having the expression for r: $r = a - bq$, we need to use $n$ (which would have similar properties) instead of $q$ since we don't know if $q$ exists yet: $r = a - bn$, where $n \in \mathbb{Z}$.\\
let's try to see what are the possible values for $r = a - bn = 27 - 5n$, when we have $n = \{... -1, 0, 1, 2, 3, 4, 5, ...\}$:\\
...\\
$n = -1 \rightarrow r = 32$\\
$n = 0 \rightarrow r = 27$\\
$n = 1 \rightarrow r = 22$\\
$n = 2 \rightarrow r = 17$\\
$n = 3 \rightarrow r = 12$\\
$n = 4 \rightarrow r = 7$\\
$n = 5 \rightarrow r = 2$\\
$n = 6 \rightarrow r = -3$\\
...\\
So values for r are a progression of numbers: $\{..., -3, 2, 7, 12, 17, 22, 27, 32, ...\}$\\
Looking at that progression we notice the reminder for division $a/b$ is the smallest non-negative member: $remainder = 2$. So using this hint, we can try to prove that $r$ exists by proving that a given set has smallest non-negative member. The Well-ordering principle states that every non-empty set of positive integers contains a least element, which might be useful.\\
Let's consider only the positive values in the progression above, namely: $\{2,7,12,17,22,27,32,...\}$, then the least element of that set is 2, which is the reminder that we are looking for. We need to construct a set for the general case and then prove that the set is non-empty, which would mean that it has least element (by WOP), which will prove that $r$ exists.

Let's consider the following set $S = \{ a - nb | a - nb \geq 0 \wedge n \in \mathbb{Z}\}$ - contains only positive integers, so we need to prove that is non-empty:\\
\textbf{Proof of} $(\forall a,b \in \mathbb{Z})(b>0)(\exists n \in \mathbb{Z})[S = \{ a - nb | a - nb \geq 0\} \textrm{ is non-empty}]$:\\
Using proof by cases:\\
1) $a \geq 0$:\\
$(\forall a,b \in \mathbb{Z})(a \geq 0)(b>0)(\exists n \in \mathbb{Z})[S = \{ a - nb | a - nb \geq 0\} \textrm{ is non-empty}]$:\\
Let $n = 0$. Then a member of the set would look like this:\\
$a - b0 \geq 0 \Leftrightarrow a \geq 0$ for each $a \geq 0$ and b, which means that the set is non-empty.\\
2) $a < 0$\\
Adding it to the expression above: $(\forall a,b \in \mathbb{Z})(a < 0)(b>0)(\exists n \in \mathbb{Z})[S = \{ a - nb | a - nb \geq 0\} \textrm{ is non-empty}]$:\\
Let $n = a$:\\
A member of the set will be: $a - bn = a - ba = a(1 - b)$\\
$1 - b \leq 0$, because $b \in \mathbb{Z} \wedge b > 0$, so\\
$a*(1-b) \geq 0$, since $a$ is negative and $1-b$ is non-positive. This shows that the set is non-empty.

By proving both cases we proved that the set $S$ is non-empty and via the WOP it has least element.\\
\\
Let's name the least element $r$. Given the definition of a member of the set we have: $r = a - bn_r$ ($n_r$ gets us the least member of the set $S$). Now it remains to prove that $0 \leq r < b$.\\
\textbf{Proof of} $0 \leq r < b$:\\
$r \geq 0$ because $r \in S$.\\
Proof of $r < b$:\\
Prove by contradiction:\\
Assume that $r \geq b$, then substituting $r$ we get:\\
$a - bn_r \geq b$\\
$a - bn_r - b \geq 0 \Leftrightarrow a - b(n_r + 1) \geq 0$, which means that $a -b(n_r + 1) \in S$, but at the same time $a - b(n_r + 1) < a - bn_r$, which means that $a - b(n_r + 1) < r$, but $r$ is the smallest element in S, which is contradiction.\\
So $r \geq b$, which concludes the proof that $0 \leq r < b$.\\

After we proved that r exists, we can continue to prove that q exists.\\

\subsubsection*{Proof of existance of q:}
From the expression $a = bq + r$, we get $q = \frac{a - r}{b}$ (1).\\
Since we proved that r exists and $r = a - bn_r$, we substitute $r$ in (1):\\
$q = \frac{a - (a-bn_r)}{b} = \frac{bn_r}{b} = n_r$, which shows that q exists.
\\
\subsection*{Proof of uniqueness of q and r:}
To prove the uniqueness we are gonna try to show that there's q' and r', but they are equal to q and r.\\
Let's assume that we can represent $a$ in two ways:\\
$a = bq + r = bq' + r'$\\
$r - r' = bq' - bq$\\
$r - r' = b(q' -q)$\\
\\
From the existance proof we know that $0 \leq r' < b$ and $0 \leq r < b$.\\
Representing $-r'$ using $0 \leq r' < b$, we get $0 \geq -r' > -b$\\
let's sum both inequalitites:\\
$-b < -r' \leq 0$ /+\\
$0 \leq r < b$\\
\\
$-b < r - r' < b$, which means that $|r - r'| < b$\\
$|b(q' - q)| < b$ (move b out of the module because $b > 0$ )\\
$b|q' - q| < b$ dividing by $b \geq 1$\\
$|q' - q| < 1$\\
this means $-1 < q' - q < 1$. Given the fact that both q' and q $\in \mathbb{N}$ the result of their subtraction should also be in $\mathbb{N}$\\
The only way for that to be possible $q' - q = 0 \equiv q' = q$, which proves the uniqueness of q and r.
\end{document}
